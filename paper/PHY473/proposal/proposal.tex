\documentclass{article}
\usepackage{amsmath}
\usepackage{amsthm}
\usepackage{amssymb}
\usepackage{graphicx}
\usepackage{hyperref}
\usepackage{enumitem}
\usepackage{geometry}
\usepackage{calligra}
\usepackage{mathpazo}
\usepackage{tikz}
\usepackage{changepage}
\usepackage{microtype}
\usepackage{nicefrac}
\usepackage{forest}
\usepackage{xcolor}
\usepackage{fontawesome5}
\usepackage{ifthen}
\usepackage[super,comma,sort&compress]{natbib}

% ==========================
% ========== meta ==========

\title{PHY473 Research Proposal: An Exploration of Quantum Algorithms for Fluid Dynamics \& Navier-Stokes}
\author{Faisal Shaik}
\date{August 2, 2025}

% === link colors ===
\hypersetup{
    colorlinks=true,
    linkcolor=blue,
    urlcolor=blue,
    citecolor=blue,
}
% ===

% === commands ===
\newcommand{\sh}[1]{\vspace{0.2cm}\textbf{#1}\vspace{-0.4cm}\\\mbox{}}
% ===

% === par indentations ===
\newenvironment{mypar}{%
    \setlength{\parskip}{1em}%
    \parindent=0pt%
}{%
    \par%
}
% ===

% === section indentations ===
\makeatletter
\renewcommand\section{\@startsection {section}{1}{\z@}%
    {-5ex \@plus -1ex \@minus -.2ex}% above
    {1ex \@plus .2ex}% below
{\normalfont\large\bfseries}}%
\makeatother
% ===

% === custom indentation ===
\newenvironment{cindent}[1]{%
    \par% start a new paragraph
    \setlength{\leftskip}{#1}% set the left indentation
    \noindent% no paragraph indentation
    \ignorespaces% ignore spaces after \begin{customindent}
}{%
    \par% end the paragraph
}
% ===

% === proof ===
\renewenvironment{proof}{\noindent{\textit{proof.}}
    \begin{cindent}{0.5cm}

    \vspace{-0.5cm}}
{\end{cindent}\hfill$\square$}
% ===

% === proof ===
\newenvironment{subproof}{\noindent{\textit{subproof.}}
    \begin{cindent}{0.5cm}

    \vspace{-0.5cm}}
{\end{cindent}\hfill$\blacksquare$}
% ===

% === thm ===
\newenvironment{thm}[2][\unskip]{%
    \vspace{0.2cm}
    \noindent{\textbf{#1} \ifthenelse{\equal{#2}{\unskip}}{}{#2}}% check if the second argument is provided
    \begin{cindent}{0.5cm}

    \vspace{-0.5cm}}
{\end{cindent}\hfill$\blacklozenge$}
% ===


% === tikz ===
\tikzset{
    every child/.style={
        edge from parent/.style={draw, ->}
    }
}
% ===

% === links ===
\newcommand{\fref}[2]{%
    \faLink\hspace{2pt}\underline{\href{#1}{\textcolor{black}{#2}}}%
}
% ===

% === other separation ===
\setlist[itemize]{topsep=0.2em}
% ===

% ========== meta ==========
% ==========================

\begin{document}

\maketitle

\section{Introduction}
\begin{mypar}
    This paper proposes the following research topic for PHY473: \textbf{an exploration on quantum and quantum-classical hybrid algorithms for solving Navier-Stokes equations efficiently}.

    \sh{Motivation:}

    The Navier-Stokes equations are the holy grail of computational fluid dynamics , governing everything from aerospace flight vehicle design , weather forecasting , plasma magneto-hydrodynamics , and astrophysics. The problem is that \textbf{these nonlinear partial differential equations are computationally intractable for classical computers} when dealing with:
    \begin{itemize}
        \item turbulent flows at high Reynolds numbers
        \item multi-scale phenomena from molecular to macroscopic scales
        \item real-time applications like weather prediction or flight control
    \end{itemize}
    A quantum approach to finding flows of a Navier–Stokes fluid may prove to be less computationally intensive , providing the possibility for potential speedups , especially when used in conjunction with classical methods.
\end{mypar}

\section{The Goals}
\begin{mypar}
    There are many things to aim for in this research topic. Here is a brief overview of goals:
    \begin{enumerate}
        \item Develop a deep understanding of fluid mechanics and the Navier-Stokes equations , both in application and derivation.
        \item Develop a deep understanding of quantum computing , and how quantum algorithms are made and used through both theory and application.
        \item Develop a good understanding on the foundations of quantum mechanics and how they let quantum computers be possible.
        \item Read existing scientific articles on the usage of quantum algorithms for solving Navier-Stokes flows and learn about these methods in depth.
        \item Develop classical and quantum solvers for Navier-Stokes equations , compare different approaches , and investigate/make new classical-quantum hybrid algorithms.
        \item Potentially publish a paper.
    \end{enumerate}
\end{mypar}

\section{Existing Works}

In addition to the works listed below , you can find many more interesting papers in the citations of said works.

\begin{mypar}
    \fref{https://www.nature.com/articles/s41534-020-00291-0}{\sh{Finding flows of a Navier–Stokes fluid through quantum computing}}

    Gaitan et al. developed a quantum algorithm that provides quadratic speedup over classical random algorithms and exponential speedup over classical deterministic algorithms for rough/non-smooth flows.

    The algorithm was validated using steady-state , inviscid compressible flow simulations in convergent divergent nozzles\cite{gaitan2020}. Such nozzles are critical components in propulsion systems , enabling the transition from subsonic to supersonic flow regimes\cite{gaitan2020}.

    For rough flows (the computationally difficult case) , there exists a regime where the speedup is quadratic over classical random algorithms and exponential over deterministic algorithms\cite{gaitan2020}.

    \fref{https://www.nature.com/articles/s42005-024-01623-8}{\sh{Quantum-inspired framework for computational fluid dynamics}}

    Peddinti, R.D., Pisoni, S., Marini, A. et al. created a revolutionary approach to computational fluid dynamics (CFD) that doesn't actually use quantum computers , but instead borrows quantum mathematical techniques to achieve exponential speedups for classical fluid simulations\cite{peddinti2024}.

    The framework is based on matrix-product states, a compressed representation of quantum states, providing memory and run time scaling logarithmically in the "mesh" size. Traditionally , CFD scales $\mathcal{O}(N^3)$ for memory and operations at every time step (when dealing with 3D flows). With this framework's scaling , they achieved $\mathcal{O}(logN)$\cite{peddinti2024}.

    \fref{https://journals.aps.org/prresearch/abstract/10.1103/PhysRevResearch.5.033182}{\sh{Quantum computing of fluid dynamics using the hydrodynamic Schrödinger equation}}

    Meng et al. present a fundamentally new approach to quantum fluid dynamics that solves the core problem of applying quantum computing to the nonlinear , non-Hamiltonian Navier-Stokes equations.

    The breakthrough is transforming fluid dynamics into quantum mechanics by using the Hydrodynamic Schrödinger Equation (HSE) , derived by generalizing the Madelung transform to compressible or incompressible flows with finite vorticity and dissipation\cite{meng2023}. Since the HSE is expressed as a unitary operator on a two-component wave function, it is more suitable than the NSE for quantum computing.

    The flow governed by the HSE can resemble a turbulent flow consisting of tangled vortex tubes with the five-thirds scaling of energy spectrum. This means the quantum formulation naturally reproduces the Kolmogorov energy cascade , which is the hallmark of turbulence\cite{meng2023} This is profound because it suggests quantum mechanics and turbulent fluid dynamics share deep mathematical structures.


    \fref{https://www.sciencedirect.com/science/article/abs/pii/S0045793024003384}{\sh{Solving incompressible Navier–Stokes via a hybrid quantum–classical scheme}}

    This paper presents the first practical demonstration of solving fluid dynamics problems on actual noisy quantum hardware , a major breakthrough in making quantum CFD a reality.

    Song et al. present a hybrid quantum–classical algorithm for the incompressible Navier–Stokes equations. A classical device performs nonlinear computations, and a quantum one uses a variational solver for the pressure Poisson equation\cite{song2024}.

    While other approaches focus on future fault-tolerant quantum computers, this paper actually implements and tests the algorithm on IBM's current noisy superconducting quantum processors.
\end{mypar}

\newpage
\bibliographystyle{unsrt}
\bibliography{references}
\end{document}
